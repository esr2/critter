\documentclass[12pt]{report}
\usepackage{geometry}                % See geometry.pdf to learn the layout options. There are lots.
\geometry{letterpaper}                   % ... or a4paper or a5paper or ... 
%\geometry{landscape}                % Activate for for rotated page geometry
%\usepackage[parfill]{parskip}    % Activate to begin paragraphs with an empty line rather than an indent
\usepackage{graphicx}
\usepackage{amssymb}
\usepackage{epstopdf}
\DeclareGraphicsRule{.tif}{png}{.png}{`convert #1 `dirname #1`/`basename #1 .tif`.png}

%code listings%%%
\usepackage{listings}
%\lstset{language=C}
\lstset{
	basicstyle=\small,
	tabsize=4
 }
%\def\lstlistlistingname{Code Excerpts}
\def\lstlistingname{Figure}
\newcommand{\refCode}{\lstlistingname \hspace{1mm}}
%%%

\usepackage{hyperref}
\usepackage{booktabs} %pretty tables

\usepackage{color}

%for title page
\newcommand{\HRule}{\rule{\linewidth}{0.5mm}}

\title{Customizable Style Checking for C Programs}
\author{Erin Rosenbaum}
%\date{}                                           % Activate to display a given date or no date

\begin{document}
%\maketitle
\begin{titlepage}

\begin{center}

\rule{\linewidth}{1.6pt}\vspace*{-\baselineskip}\vspace*{2pt}
\rule{\linewidth}{0.4pt}\\[\baselineskip]
{\huge CritTer \\[0.5\baselineskip] Critique From the Terminal}\\[0.2\baselineskip]
\rule{\linewidth}{0.4pt}\vspace*{-\baselineskip}\vspace{3.2pt}
\rule{\linewidth}{1.6pt}\\[\baselineskip]
\scshape
\Large Customizable Style Checking for C Programs \par
\vspace*{4\baselineskip}

% Author and supervisor
\begin{minipage}{0.4\textwidth}
\begin{flushleft} \large
\emph{Author:}\\
Erin \textsc{Rosenbaum}
\end{flushleft}
\end{minipage}
\begin{minipage}{0.4\textwidth}
\begin{flushright} \large
\emph{Advisor:} \\
Dr.~Robert \textsc{Dondero}
\end{flushright}
\end{minipage}

\vspace{5cm}
\small{
Senior Thesis \\[0.4cm]
Submitted to Princeton University \\
Department of Computer Science\\
In Partial Fulfillment of the Requirements for the A.B. Degree \\
}

\vfill

% Bottom of the page
{\large \today}

\end{center}

\end{titlepage}


\begin{abstract}
\end{abstract}

\tableofcontents

\listoftables
\listoffigures

\chapter{Problem Description}

\begin{table}%[h]
	\caption{Error Checking}
	\label{errorChecking}
	\begin{center}
	\begin{tabular}{ccc}
		\toprule
		Types of Errors & Tools for Literature & Tools for Code \\
		\midrule
		Syntactic & Spell Check & Compiler \\
		Semantic & Grammar Check & \\ 
		Substantive & Audience/Grader & User/Tests \\
		\bottomrule
	\end{tabular}
	\end{center}
\end{table}

When writing, one encounters three different types of errors: syntactic, semantic and substantive.  In 
the case of writing literature, these errors take the form of spelling, grammar and logical mistakes. 
When writing code, they form syntax errors, poorly written and malfunctioning code. Both spelling 
mistakes and syntax errors represent text that contains something outside the language (be it 
English, C, etc.). A passage with bad grammar and poorly written code both represent text that is 
technically valid but hard to understand. Finally, illogical arguments and malfunctioning code both 
represent errors in the ideas behind the text. 
There are tools to help one find these errors in both literature and code (see Table \ref
{errorChecking}). Spelling and Grammar Check exist on nearly every text editor. The reader highlights 
substantive errors in literature. With code, the compiler shows semantic errors and the user/testing 
reveal substantive errors. Yet the are of semantic error checking for code has yet to be automated 
fully.

Seen from a different point of view, this problem can be formulated in terms of software quality. 
Namely, there are two perspectives on software quality: that of the user and that of the programmer. 
Users care about whether or not a piece of software works (i.e.\ behaves as it ought to). In contrast, 
programmers care about if software is easily maintainable. Minimally, maintainability implies that 
code is easy to read and update. Evaluating the user�s perspective of a program, is possible and most 
easily accomplished through automated testing. Though it is possible to evaluate the programmer's 
perspective, tools to so only check for certain qualities. Given that code quality is 
subjective, any tool that only performs pre-defined inspections will never be satisfactory to every 
programmer.

While I have talked about the nature of semantic error checking, I have yet to comment on why it is 
important. The biggest reason to perform semantic error checking is to improve readability (the 
ease with which another programmer can understand a piece of code). In the same way that 
grammatical errors in a paper can confound its underlying arguments, poorly written code can easily 
obscure its underlying function. Similarly, readable code is easy to revise and update later.  In an 
industry or academic setting where other individuals will necessarily need to read or edit one's code, 
maintaining readability is essential. In an academic environment, automated semantic error checking 
can immediately save work for professors by replacing the process of writing the same set of stylistic comments to many students with a set of automated warnings.
Students can also directly benefit by applying this tool to their code before submitting assignments --- 
giving them the chance to improve their grades as well as their coding habits.

\chapter{Related Products}

There are some tools that try to fill the semantic error checking gap. Each approaches the 
problem differently, but all help to find some semantic errors. The tools that I will be discussing are 
Splint \cite{splint-manual} , PMD \cite{pmd}, and Checkstyle \cite{checkstyle}.

\section{Splint}
Splint is a tool for ``statically checking C programs for security vulnerabilities and programming 
mistakes" \cite[p.\ 9]{splint-manual}. It works exactly as my program will from the student�s 
perspective, i.e.\ as a command-line program which prints warnings and errors to \lstinline{stdout}. 
{\color{red} Without additional information}, Splint displays warnings about basic semantic errors like 
assignments with mismatched types and ignored return values. With some more effort, 
programmers can add annotations (i.e.\ fancy comments) that give Splint a specification to 
check against. These annotations allow for stronger checks like memory management, null 
pointers and ``violations of information hiding" \cite[p. 9]{splint-manual}. An examples of annotations 
in action are the following abstract type declarations (\refCode \ref{splint-annotations}). These 
annotations define \lstinline{mstring} and \lstinline{weekDay} as abstract types and further 
specify that they are mutable/immutable respectively.
%
\begin{figure}[h]
\caption{Splint Annotations}
\label{splint-annotations}
\begin{lstlisting}[frame=single, language=C]
typedef /*@abstract@*/ /*@mutable@*/ char *mstring;
typedef /*@abstract@*/ /*@immutable@*/ int weekDay;
\end{lstlisting}
\end{figure}

While these annotations provide an extensive feature set, they are a hugely inconvenient. They 
require programmers to specifically write their code to meet the specification, not only of the client, but 
also of the tool. For new programmers (often the ones who need the most error checking), these 
annotations are almost impossible to implement on top of learning to {\color{red}program}. As 
described by David Evans (one of the authors) in an email, ``One of the goals of the original design of 
Splint was for programmers who add no annotations to start getting some useful warnings right away, 
including warnings that encourage them to start adding annotations.  For some aspects, such as 
\lstinline{/*@null@*/} annotations I think this has worked okay, but for others like abstract types, 
memory management, etc. I don't think it has worked very well, and the warnings on these issues tend 
to either make developers want to stop using Splint, or at least just turn off all the warnings of that type, 
rather than start adding the annotations needed to enable better checking." \cite{evans-email} 

\section{PMD and Checkstyle}
PMD is a tool for checking Java Code that is integrated into a dozen or so different popular IDEs.
PMD comes premade with over 250 checks broken up mostly by purpose such as Braces Rules, 
Basic Rules, Coupling Rules, etc. Some of these checks also deal explicitly with a certain 
library or platform like Android, Jakarta and JUnit. PMD works by passing source code into a 
JavaCC-generated parser and receiving an Abstract Syntax Tree (a.k.a.\ AST, a tree-based model of 
the source code). PMD then traverses the AST and calls each rule in the 
RuleSet to check for any violations. This pattern of examining a tree of nodes is called the Visitor 
Pattern \cite{design-patterns}. The RuleSet is an XML file that can be edited to augment the 
functionality of PMD with customized rules. Rules are written in their own classes and extend a 
base implementation. The rule itself can override three functions (start, visit and end) to perform 
various checks against the source code based on the nodes in the AST. Here is the ``dummy" 
example from the PMD website which counts how many expressions are in the source code 
(\refCode \ref{pmd-rule}).\cite{pmd}

\begin{figure}
\caption[Example PMD Rule]{Example PMD Rule counting the number expressions}
\label{pmd-rule}
\begin{lstlisting}[frame=single, language=Java]
package net.sourceforge.pmd.rules;

import java.util.concurrent.atomic.AtomicLong;
import net.sourceforge.pmd.AbstractJavaRule;
import net.sourceforge.pmd.RuleContext;
import net.sourceforge.pmd.ast.ASTExpression;

public class CountRule extends AbstractJavaRule {

	private static final String COUNT = ``count'';

	@Override
	public void start(RuleContext ctx) {
		ctx.setAttribute(COUNT, new AtomicLong());
		super.start(ctx);
	}

	@Override
	public Object visit(ASTExpression node, Object data) {
		// How many Expression nodes are there in all
		// files parsed!  I must know!
		RuleContext ctx = (RuleContext)data;
		AtomicLong total = 
			(AtomicLong)ctx.getAttribute(COUNT);
		total.incrementAndGet();
		return super.visit(node, data);
	}

	@Override
	public void end(RuleContext ctx) {
		AtomicLong total = 
			(AtomicLong)ctx.getAttribute(COUNT);
		addViolation(ctx, null, new Object[] { total });
		ctx.removeAttribute(COUNT);
		super.start(ctx);
	}
}
\end{lstlisting}
\end{figure}

Checkstyle mimics PMD in function but provides different checks out of the box (namely those 
regarding duplicate code, class design, whitespace, etc). It also uses the Visitor Pattern and 
provides ways to examine the beginning and ends of the tree as well as each node. New rules are 
added through an XML file and similarly passed to Checkstyle at runtime.\cite{checkstyle}


\nocite{framaC}


\chapter{Specifications/High Level Overview of Program}

\section{Functional}

\section{Requirement}


\nocite{*}
\bibliographystyle{plain}
\bibliography{Bibliography}

\end{document}  